\documentclass[12pt]{article}

\usepackage{times, amsmath, amssymb, latexsym, amsthm}
\usepackage{setspace} 

\title{HOMEWORK \#1}
\author{Ann Guilinger \\ \and Raphael Leu \\ \and Brian Roberts}

\begin{document}
\maketitle

\section{Break Math}

See Ruby file hw1-1.rb

\section{Matrix inversion}

See Ruby file hw1-2.rb

We used the Ruby matrix library.  This inverter uses Gauss-Jordan with pivoting.

\section{Binary and Floating point}

\hspace{-.7in}\begin{tabular}{c | c | c}
Decimal & Binary & Floating point\\ \hline
10.5 & 1010.1 & 0 10000000010 10101000000000000000000000000000000000000000000000000  \\
$\frac{1}{3}$ & $\overline{.01}$ & 0 01111111101 10101010101010101010101010101010101010101010101010101 \\
$\frac{22}{7}$ & $11\overline{.001}$ & 0 10000000000 11001001001001001001001001001001001001001000100000111
\end{tabular}

\pagebreak

\section{Head-to-head}

\begin{enumerate}
\item Bisection for $f(x)=x^4-2$

\begin{footnotesize}
\begin{spacing}{-.8}
xmid: 1.0, y value: -1.0\\
xmid: 1.0, y value: 3.0625\\
xmid: 1.0, y value: 0.44140625\\
xmid: 1.125, y value: -0.398193359375\\
xmid: 1.1875, y value: -0.0114593505859375\\
xmid: 1.1875, y value: 0.2062692642211914\\
xmid: 1.1875, y value: 0.09528452157974243\\
xmid: 1.1875, y value: 0.041389349848032\\
xmid: 1.1875, y value: 0.014835044974461198\\
xmid: 1.1875, y value: 0.0016554650064790621\\
xmid: 1.1884765625, y value: -0.004910025046228839\\
xmid: 1.18896484375, y value: -0.0016293022444529015\\
xmid: 1.18896484375, y value: 1.2575617237331471e-05\\
xmid: 1.1890869140625, y value: -0.0008084897285944859\\
xmid: 1.18914794921875, y value: -0.00039798866266971267\\
xmid: 1.189178466796875, y value: -0.00019271442486945567\\
xmid: 1.1891937255859375, y value: -9.007137940519883e-05\\
xmid: 1.1892013549804688, y value: -3.8748374987518375e-05\\
xmid: 1.1892051696777344, y value: -1.3086502351766782e-05\\
xmid: 1.1892070770263672, y value: -2.554734264137437e-07\\
xmid: 1.1892070770263672, y value: 6.160064188076575e-06\\
xmid: 1.1892070770263672, y value: 2.9522934514858434e-06\\
xmid: 1.1892070770263672, y value: 1.348409530255168e-06\\
xmid: 1.1892070770263672, y value: 5.464679313504917e-07\\
xmid: 1.1892070770263672, y value: 1.454972222703077e-07\\
xmid: 1.1892071068286896, y value: -5.498810962123457e-08\\
xmid: 1.1892071068286896, y value: 4.525455432613512e-08\\
xmid: 1.1892071142792702, y value: -4.866778091638935e-09\\
xmid: 1.1892071142792702, y value: 2.019388789520349e-08\\
xmid: 1.1892071142792702, y value: 7.663555123826882e-09\\
xmid: 1.1892071142792702, y value: 1.3983885160939735e-09\\
xmid: 1.1892071147449315, y value: -1.7341947877724806e-09\\
xmid: 1.189207114977762, y value: -1.67903246861556e-10\\
Final y value: -1.67903246861556e-10\\
Final x value: 1.1892071152105927\\
Number of iterations: 33\\
\end{spacing}
\end{footnotesize}
\vspace{.4in}
The bisection produced results within 8 digits of accuracy in 33 iterations.  The convergence in the y-axis seemed to waiver between positive and negative y-values at random.

\pagebreak
\item Secant for $f(x)=x^4-2$

\begin{footnotesize}
\begin{spacing}{-.8}
xold: 2.0, xnew: 0.25, y value: -1.99609375\\
xold: 0.25, xnew: 0.4683760683760684, y value: -1.9518741085724152\\
xold: 0.4683760683760684, xnew: 10.107590228289883, y value: 10435.356258593585\\
xold: 10.107590228289883, xnew: 0.47017869145727964, y value: -1.9511289385357296\\
xold: 0.47017869145727964, xnew: 0.4719802894982221, y value: -1.9503755793809\\
xold: 0.4719802894982221, xnew: 5.1361471750910574, y value: 693.904811749434\\
xold: 5.1361471750910574, xnew: 0.48505323536033096, y value: -1.9446449021369268\\
xold: 0.48505323536033096, xnew: 0.4980513429722569, y value: -1.9384686474022659\\
xold: 0.4980513429722569, xnew: 4.577614956156458, y value: 437.0935297596889\\
xold: 4.577614956156458, xnew: 0.5160639392931019, y value: -1.9290725979414054\\
xold: 0.5160639392931019, xnew: 0.5339104618404529, y value: -1.9187405790343723\\
xold: 0.5339104618404529, xnew: 3.848155946132483, y value: 217.28587285784266\\
xold: 3.848155946132483, xnew: 0.5629206966155693, y value: -1.899587248960527\\
xold: 0.5629206966155693, xnew: 0.5913924345984497, y value: -1.8776784267907698\\
xold: 0.5913924345984497, xnew: 3.031540415601771, y value: 82.4604614727347\\
xold: 3.031540415601771, xnew: 0.6457191393083876, y value: -1.8261500232815193\\
xold: 0.6457191393083876, xnew: 0.6974102383087862, y value: -1.7634334834441627\\
xold: 0.6974102383087862, xnew: 2.150835682533229, y value: 19.400746929765756\\
xold: 2.150835682533229, xnew: 0.8185119792689675, y value: -1.5511511058682408\\
xold: 0.8185119792689675, xnew: 0.9171491233681796, y value: -1.292445624208415\\
xold: 0.9171491233681796, xnew: 1.4099223899318556, y value: 1.951671449155835\\
xold: 1.4099223899318556, xnew: 1.113468379946871, y value: -0.46286661840229004\\
xold: 1.113468379946871, xnew: 1.1702985732485325, y value: -0.12419926249858793\\
xold: 1.1702985732485325, xnew: 1.191139873635705, y value: 0.01303373004931263\\
xold: 1.191139873635705, xnew: 1.1891604670944667, y value: -0.00031379000692033365\\
xold: 1.1891604670944667, xnew: 1.1892070014311567, y value: -7.640152617227614e-07\\
xold: 1.1892070014311567, xnew: 1.1892071150094037, y value: 4.495515071312184e-11\\
Final y value: 4.495515071312184e-11\\
Final x value: 1.1892071150094037\\
Number of iterations: 27\\
\end{spacing}
\end{footnotesize}
\vspace{.4in}
The secant method produced results in 27 iterations.  The convergence has a pattern of very large positive values followed by two small negative values.  This makes sense considering the root was bracketed, so each large guess will lead to a btter, smaller guess in the next iteration.

\pagebreak
\item False position for $f(x)=x^4-2$

\begin{footnotesize}
\begin{spacing}{-.8}
lower bound: 0.25, upper bound: 2.0, y value: -1.99609375\\
lower bound: 0.4683760683760684, upper bound: 2.0, y value: -1.9518741085724152\\
lower bound: 0.6557858407864514, upper bound: 2.0, y value: -1.815052637648054\\
lower bound: 0.8100578189546664, upper bound: 2.0, y value: -1.5694098673868484\\
lower bound: 0.9300050113311887, upper bound: 2.0, y value: -1.251931866272392\\
lower bound: 1.0178339391556377, upper bound: 2.0, y value: -0.9267331576136155\\
lower bound: 1.0788121749997586, upper bound: 2.0, y value: -0.6454864426247353\\
lower bound: 1.1194126872789572, upper bound: 2.0, y value: -0.4297785731256021\\
lower bound: 1.1456402247879947, upper bound: 2.0, y value: -0.2773659977159153\\
lower bound: 1.1622378487123188, upper bound: 2.0, y value: -0.1753479592382161\\
lower bound: 1.1726009018082801, upper bound: 2.0, y value: -0.10939459391485573\\
lower bound: 1.1790159884195255, upper bound: 2.0, y value: -0.06768120516830023\\
lower bound: 1.1829658353429302, upper bound: 2.0, y value: -0.04165678048632615\\
lower bound: 1.1853896955311558, upper bound: 2.0, y value: -0.02555704581654239\\
lower bound: 1.1868740595964067, upper bound: 2.0, y value: -0.015648737161840565\\
lower bound: 1.1877819301031147, upper bound: 2.0, y value: -0.009570241854789385\\
lower bound: 1.188336773915848, upper bound: 2.0, y value: -0.005848509189096074\\
lower bound: 1.188675705172536, upper bound: 2.0, y value: -0.0035724894706565813\\
lower bound: 1.1888826843203741, upper bound: 2.0, y value: -0.002181607824133902\\
lower bound: 1.1890090603331798, upper bound: 2.0, y value: -0.0013320148893902761\\
lower bound: 1.18908621385012, upper bound: 2.0, y value: -0.0008131987448745104\\
lower bound: 1.1891333135480973, upper bound: 2.0, y value: -0.0004964288145836715\\
lower bound: 1.1891620652134391, upper bound: 2.0, y value: -0.00030304043021711635\\
lower bound: 1.189179616024724, upper bound: 2.0, y value: -0.00018498391974808293\\
lower bound: 1.1891903293640844, upper bound: 2.0, y value: -0.0001129174760878815\\
lower bound: 1.1891968689243104, upper bound: 2.0, y value: -6.892623404963771e-05\\
lower bound: 1.1892008607336848, upper bound: 2.0, y value: -4.207320739291198e-05\\
lower bound: 1.1892032973635294, upper bound: 2.0, y value: -2.568178922102149e-05\\
lower bound: 1.189204784697231, upper bound: 2.0, y value: -1.567631818732984e-05\\
lower bound: 1.1892056925736272, upper bound: 2.0, y value: -9.56890703962543e-06\\
lower bound: 1.1892062467457738, upper bound: 2.0, y value: -5.840906839083004e-06\\
lower bound: 1.189206585014974, upper bound: 2.0, y value: -3.56531598955101e-06\\
lower bound: 1.1892067914959734, upper bound: 2.0, y value: -2.1762844273354176e-06\\
lower bound: 1.1892069175328561, upper bound: 2.0, y value: -1.328413281598273e-06\\
lower bound: 1.1892069944662989, upper bound: 2.0, y value: -8.10869039469253e-07\\
lower bound: 1.1892070414267923, upper bound: 2.0, y value: -4.949578318313996e-07\\
lower bound: 1.1892070700916717, upper bound: 2.0, y value: -3.0212430623954845e-07\\
lower bound: 1.189207087588832, upper bound: 2.0, y value: -1.8441792137835478e-07\\
lower bound: 1.1892070982691707, upper bound: 2.0, y value: -1.1256945775528493e-07\\
lower bound: 1.1892071047884933, upper bound: 2.0, y value: -6.871285895826418e-08\\
lower bound: 1.1892071087679146, upper bound: 2.0, y value: -4.19426111619714e-08\\
lower bound: 1.1892071111969695, upper bound: 2.0, y value: -2.5601942343911333e-08\\
lower bound: 1.1892071126796748, upper bound: 2.0, y value: -1.5627529936779183e-08\\
lower bound: 1.1892071135847242, upper bound: 2.0, y value: -9.539107947986736e-09\\
lower bound: 1.18920711413717, upper bound: 2.0, y value: -5.8227098698182544e-09\\
lower bound: 1.1892071144743852, upper bound: 2.0, y value: -3.554205818545597e-09\\
lower bound: 1.1892071146802228, upper bound: 2.0, y value: -2.1695012453193385e-09\\
lower bound: 1.1892071148058667, upper bound: 2.0, y value: -1.324272913549862e-09\\
lower bound: 1.1892071148825603, upper bound: 2.0, y value: -8.083422819993302e-10\\
Final y value: -8.083422819993302e-10\\
Final x value: 1.1892071148825603\\
Number of iterations: 49\\
\end{spacing}
\end{footnotesize}
\vspace{.4in}

This method acheives 8 digits of accuracy in 49 iterations.  This method converges with each step getting closer at a constant rate (although it is consistently negative which is logical considering the method keeps the intial guess of 2 throughout the algorithm).

\pagebreak
\item FPI for $g(x)=\frac{x}{2}+\frac{1}{x^3}$

\begin{footnotesize}
\begin{spacing}{-.8}
old x: 1.0, new x: 1.5, difference: 0.5\\
old x: 1.5, new x: 1.0462962962962963, difference: -0.4537037037037037\\
old x: 1.0462962962962963, new x: 1.3961917541713085, difference: 0.3498954578750122\\
old x: 1.3961917541713085, new x: 1.065517569872507, difference: -0.33067418429880147\\
old x: 1.065517569872507, new x: 1.3594020918672058, difference: 0.29388452199469883\\
old x: 1.3594020918672058, new x: 1.077768066601146, difference: -0.28163402526605985\\
old x: 1.077768066601146, new x: 1.3376582996965998, difference: 0.2598902330954538\\
old x: 1.3376582996965998, new x: 1.0866253005767548, difference: -0.251032999119845\\
old x: 1.0866253005767548, new x: 1.3227129378725515, difference: 0.23608763729579674\\
old x: 1.3227129378725515, new x: 1.0934753027729127, difference: -0.2292376350996388\\
old x: 1.0934753027729127, new x: 1.3115820042343793, difference: 0.2181067014614666\\
old x: 1.3115820042343793, new x: 1.0990051879662324, difference: -0.21257681626814695\\
old x: 1.0990051879662324, new x: 1.3028594974358345, difference: 0.20385430946960215\\
old x: 1.3028594974358345, new x: 1.1036054808456455, difference: -0.19925401659018904\\
old x: 1.1036054808456455, new x: 1.2957779322327534, difference: 0.19217245138710792\\
old x: 1.2957779322327534, new x: 1.1075188544259187, difference: -0.1882590778068347\\
old x: 1.1075188544259187, new x: 1.2898760327884606, difference: 0.18235717836254195\\
old x: 1.2898760327884606, new x: 1.1109060030282172, difference: -0.17897002976024345\\
old x: 1.1109060030282172, new x: 1.284856864848635, difference: 0.17395086182041775\\
old x: 1.284856864848635, new x: 1.113878554204667, difference: -0.17097831064396796\\
old x: 1.113878554204667, new x: 1.2805191411018542, difference: 0.1666405868971872\\
old x: 1.2805191411018542, new x: 1.116517013665047, difference: -0.1640021274368071\\
old x: 1.116517013665047, new x: 1.2767207734536554, difference: 0.16020375978860835\\
old x: 1.2767207734536554, new x: 1.1188812244487978, difference: -0.15783954900485764\\
old x: 1.1188812244487978, new x: 1.2733581347570162, difference: 0.15447691030821842\\
old x: 1.2733581347570162, new x: 1.121016793695299, difference: -0.15234134106171715\\
old x: 1.121016793695299, new x: 1.2703535868879654, difference: 0.14933679319266635\\
old x: 1.2703535868879654, new x: 1.1229592151180952, difference: -0.14739437176987025\\
old x: 1.1229592151180952, new x: 1.2676476340732572, difference: 0.14468841895516205\\
old x: 1.2676476340732572, new x: 1.1247366095948568, difference: -0.14291102447840043\\
old x: 1.1247366095948568, new x: 1.2651937974661023, difference: 0.1404571878712455\\
old x: 1.2651937974661023, new x: 1.126371602994749, difference: -0.13882219447135324\\
old x: 1.126371602994749, new x: 1.2629551593016917, difference: 0.13658355630694263\\
old x: 1.2629551593016917, new x: 1.1278826466348442, difference: -0.1350725126668475\\
old x: 1.1278826466348442, new x: 1.2609019684937854, difference: 0.13301932185894128\\
old x: 1.2609019684937854, new x: 1.1292849669246898, difference: -0.13161700156909562\\
old x: 1.1292849669246898, new x: 1.2590099420143555, difference: 0.12972497508966563\\
old x: 1.2590099420143555, new x: 1.1305912619299407, difference: -0.1284186800844147\\
old x: 1.1305912619299407, new x: 1.2572590346696955, difference: 0.1266677727397547\\
old x: 1.2572590346696955, new x: 1.13181222128628, difference: -0.12544681338341546\\
old x: 1.13181222128628, new x: 1.255632531659297, difference: 0.12382031037301688\\
old x: 1.255632531659297, new x: 1.1329569203361833, difference: -0.12267561132311355\\
old x: 1.1329569203361833, new x: 1.2541163682283913, difference: 0.121159447892208\\
...
\end{spacing}
\end{footnotesize}
\vspace{.4in}
The error in the y-axis switches off between being positive and negative for this method, although the backward error got consistently smaller, the method never converged to be within 8 digits of accuracy.  This makes sense, as linear convergence is not predicted since $g'(x)=\frac{1}{2} - 3\frac{1}{x^4}$, so $g'(\sqrt[4]{2})\approx -2.02269$, therefore $|g'(r)|\geq 1$, so linear convergence will not exist in this case.

\pagebreak
\item FPI for $g(x)=\frac{2x}{3}+\frac{2}{3x^3}$

\begin{footnotesize}
\begin{spacing}{-.8}
old x: 1.0, new x: 1.3333333333333333, difference: 0.33333333333333326\\
old x: 1.3333333333333333, new x: 1.1701388888888888, difference: -0.16319444444444442\\
old x: 1.1701388888888888, new x: 1.1961914292767475, difference: 0.026052540387858647\\
old x: 1.1961914292767475, new x: 1.1869602531658547, difference: -0.009231176110892747\\
old x: 1.1869602531658547, new x: 1.1899645861030304, difference: 0.0030043329371756133\\
old x: 1.1899645861030304, new x: 1.1889555885621519, difference: -0.0010089975408784646\\
old x: 1.1889555885621519, new x: 1.1892910635866454, difference: 0.0003354750244934923\\
old x: 1.1892910635866454, new x: 1.1891791439922261, difference: -0.00011191959441925192\\
old x: 1.1891791439922261, new x: 1.1892164399887344, difference: 3.7295996508257545e-05\\
old x: 1.1892164399887344, new x: 1.189204006820289, difference: -1.2433168445458165e-05\\
old x: 1.189204006820289, new x: 1.1892081510797794, difference: 4.144259490468372e-06\\
old x: 1.1892081510797794, new x: 1.1892067696455069, difference: -1.3814342725293471e-06\\
old x: 1.1892067696455069, new x: 1.1892072301219931, difference: 4.6047648627478566e-07\\
old x: 1.1892072301219931, new x: 1.1892070766296527, difference: -1.5349234039341297e-07\\
old x: 1.1892070766296527, new x: 1.1892071277937464, difference: 5.1164093628486285e-08\\
old x: 1.1892071277937464, new x: 1.1892071107390463, difference: -1.7054700096608144e-08\\
old x: 1.1892071107390463, new x: 1.1892071164239462, difference: 5.684899884172978e-09\\
old x: 1.1892071164239462, new x: 1.1892071145289793, difference: -1.8949668501022643e-09\\
old x: 1.1892071145289793, new x: 1.189207115160635, difference: 6.316556167007548e-10\\
Final y value: 6.316556167007548e-10\\
Final x value: 1.189207115160635\\
Number of iterations: 19\\
\end{spacing}
\end{footnotesize}
\vspace{.4in}

This method converges to 8 digits of accuracy within 19 iterations.  The method produced y-values switching off from positive to negative.  The quick convergence implies that the equation was well built with $\alpha$, since for $\frac{3}{2}g_\alpha(x)=x+\frac{1}{x^3}$, so $f(x)=\frac{1}{x^3}$, so $f'(r)=\frac{-3}{2}$, therefore $\alpha=\frac{2}{3}$.

\pagebreak
\item FPI for $g(x)=x-\frac{2}{5}(x^4-2)$

\begin{footnotesize}
\begin{spacing}{-.8}
old x: 1.0, new x: 1.15, difference: 0.1499999999999999\\
old x: 1.15, new x: 1.1876490625, difference: 0.03764906250000011\\
old x: 1.1876490625, new x: 1.1892181683767857, difference: 0.0015691058767857147\\
old x: 1.1892181683767857, new x: 1.1892070145301257, difference: -1.1153846660061362e-05\\
old x: 1.1892070145301257, new x: 1.1892071159145672, difference: 1.0138444150875614e-07\\
old x: 1.1892071159145672, new x: 1.1892071149944454, difference: -9.201217565646402e-10\\
old x: 1.1892071149944454, new x: 1.189207115002796, difference: 8.350653502020577e-12\\
Final y value: 8.350653502020577e-12\\
Final x value: 1.189207115002796\\
Number of iterations: 7\\
\end{spacing}
\end{footnotesize}
\vspace{.4in}
This method yielded the quickest convergence, with the method getting to 8 digits of accuracy within 7 iterations.  The second guess got within nearly 0.15 of the origin, and quickly got closer.  This quick convergence follows logically from picking our $\beta$ to be close to $-g'(r)$ since $-g'(r) = 8/5 = 1.6$ while $-g'(r) \approx 1.6908685$.

\end{enumerate}

\pagebreak
\section{More FPI}
\begin{enumerate}
\item $g(x)= \frac{2x^3-1}{6}$

First test:
\begin{footnotesize}
\begin{spacing}{-.8}
old x: 1.0, new x: 0.16666666666666666, difference: -0.8333333333333334\\
old x: 0.16666666666666666, new x: -0.16512345679012344, difference: -0.3317901234567901\\
old x: -0.16512345679012344, new x: -0.16816740529326554, difference: -0.003043948503142102\\
old x: -0.16816740529326554, new x: -0.16825194022334902, difference: -8.453493008347968e-05\\
old x: -0.16825194022334902, new x: -0.1682543320964722, difference: -2.3918731231753476e-06\\
old x: -0.1682543320964722, new x: -0.16825439980829027, difference: -6.77118180691938e-08\\
old x: -0.16825439980829027, new x: -0.16825440172518014, difference: -1.916889869058025e-09\\
old x: -0.16825440172518014, new x: -0.1682544017794464, difference: -5.426625815374564e-11\\
Final y value: -5.426625815374564e-11\\
Final x value: -0.1682544017794464\\
Number of iterations: 8\\
\end{spacing}
\end{footnotesize}

\vspace{.25in}
Second test:
\vspace{.05in}

\begin{footnotesize}
\begin{spacing}{-.8}
old x: -1.0, new x: -0.5, difference: 0.5\\
old x: -0.5, new x: -0.20833333333333334, difference: 0.29166666666666663\\
old x: -0.20833333333333334, new x: -0.16968074845679013, difference: 0.03865258487654322\\
old x: -0.16968074845679013, new x: -0.1682951242795515, difference: 0.0013856241772386146\\
old x: -0.1682951242795515, new x: -0.16825555489542218, difference: 3.9569384129328755e-05\\
old x: -0.16825555489542218, new x: -0.16825443442539353, difference: 1.120470028648235e-06\\
old x: -0.16825443442539353, new x: -0.1682544027051747, difference: 3.1720218840458614e-08\\
old x: -0.1682544027051747, new x: -0.16825440180718962, difference: 8.979850751877905e-10\\
Final y value: 8.979850751877905e-10\\
Final x value: -0.16825440180718962\\
Number of iterations: 8\\
\end{spacing}
\end{footnotesize}
\vspace{.25in}

Notice that this function can only find the root at $-0.16825440180718962$, even though our intial guess were near the other two roots.  This implies that the function we built for FPI only can capture the root for which the function is going from positive to negative.  Near this root, though, the function is very stable, since it found the root within 8 iterations.

\item $g(x) = x-\frac{2x^3-6x-1}{6x^2-6}$

\begin{footnotesize}
\begin{spacing}{-.8}
old x: -2.0, new x: -1.7222222222222223, difference: 0.2777777777777777\\
old x: -1.7222222222222223, new x: -1.6473632187917904, difference: 0.07485900343043195\\
old x: -1.6473632187917904, new x: -1.6418134196310956, difference: 0.005549799160694757\\
old x: -1.6418134196310956, new x: -1.6417835283181434, difference: 2.9891312952168292e-05\\
old x: -1.6417835283181434, new x: -1.6417835274529258, difference: 8.652176752832474e-10\\
Final y value: 8.652176752832474e-10\\
Final x value: -1.6417835274529258\\
Number of iterations: 5\\
\end{spacing}
\end{footnotesize}
\vspace{.4in}

\begin{footnotesize}
\begin{spacing}{-.8}
old x: -0.5, new x: -0.1111111111111111, difference: 0.3888888888888889\\
old x: -0.1111111111111111, new x: -0.16782407407407407, difference: -0.056712962962962965\\
old x: -0.16782407407407407, new x: -0.1682543697750313, difference: -0.0004302957009572339\\
old x: -0.1682543697750313, new x: -0.16825440178102724, difference: -3.200599593689013e-08\\
Final y value: -3.200599593689013e-08\\
Final x value: -0.16825440178102724\\
Number of iterations: 4\\
\end{spacing}
\end{footnotesize}
\vspace{.4in}

\begin{footnotesize}
\begin{spacing}{-.8}
old x: 2.0, new x: 1.8333333333333333, difference: -0.16666666666666674\\
old x: 1.8333333333333333, new x: 1.8104575163398693, difference: -0.02287581699346397\\
old x: 1.8104575163398693, new x: 1.8100380691578124, difference: -0.0004194471820568868\\
old x: 1.8100380691578124, new x: 1.8100379292339688, difference: -1.399238436228245e-07\\
old x: 1.8100379292339688, new x: 1.810037929233953, difference: -1.5765166949677223e-14\\
Final y value: -1.5765166949677223e-14\\
Final x value: 1.810037929233953\\
Number of iterations: 5\\
\end{spacing}
\end{footnotesize}
\vspace{.4in}
Yay it found all the roots!\\
This equation found all the roots with an initial guess near the actual value of the root.  Being near the root is important, as our function was built assuming that there is a neighborhood of r such that this function converges.  This yeilds very quick results for all three roots, though, so as long as our initial guesses can be close enough, this method is very effective.

\item $g(x) = x^4 - 3x^2 + \frac{x}{2}$

\begin{footnotesize}
\begin{spacing}{-.8}
old x: -1.0, new x: -2.5, difference: -1.5\\
old x: -2.5, new x: 19.0625, difference: 21.5625\\
old x: 19.0625, new x: 130963.62403869629, difference: 130944.56153869629\\
old x: 130963.62403869629, new x: 2.941729512841314e+20, difference: 2.9417295128413127e+20\\
old x: 2.941729512841314e+20, new x: 7.488777894424145e+81, difference: 7.488777894424145e+81\\
old x: 7.488777894424145e+81, new x: Infinity, difference: Infinity\\
old x: Infinity, new x: NaN, difference: NaN\\
Final y value: NaN
Final x value: NaN
\end{spacing}
\end{footnotesize}

\vspace{.4in}

\begin{footnotesize}
\begin{spacing}{-.8}
old x: 2.0, new x: 5.0, difference: 3.0\\
old x: 5.0, new x: 552.5, difference: 547.5\\
old x: 552.5, new x: 93180462671.5625, difference: 93180462119.0625\\
old x: 93180462671.5625, new x: 7.538751886004188e+43, difference: 7.538751886004188e+43\\
old x: 7.538751886004188e+43, new x: 3.2299648823842375e+175, difference: 3.2299648823842375e+175\\
old x: 3.2299648823842375e+175, new x: NaN, difference: NaN\\
Final y value: NaN
Final x value: NaN
Number of iterations: 6
\end{spacing}
\end{footnotesize}

\vspace{.4in}


\begin{footnotesize}
\begin{spacing}{-.8}
old x: -0.5, new x: -0.9375, difference: -0.4375\\
old x: -0.9375, new x: -2.3329925537109375, difference: -1.3954925537109375\\
old x: -2.3329925537109375, new x: 12.129603404604936, difference: 14.462595958315873\\
old x: 12.129603404604936, new x: 21211.11911870733, difference: 21198.989515302725\\
old x: 21211.11911870733, new x: 2.0242042331057843e+17, difference: 2.0242042331055722e+17\\
old x: 2.0242042331057843e+17, new x: 1.6788709519617318e+69, difference: 1.6788709519617318e+69\\
old x: 1.6788709519617318e+69, new x: 7.944549216216155e+276, difference: 7.944549216216155e+276\\
old x: 7.944549216216155e+276, new x: NaN, difference: NaN\\
Final y value: NaN
Final x value: NaN
Number of iterations: 8
\end{spacing}
\end{footnotesize}
\vspace{.4in}

This method found none of the roots.  In fact, for each initial guess, the function diverged to values too large for the computer to handle within 6-8 iterations.  This was completely unstable.

\end{enumerate}

\section{Secants behaving badly}

The slope of the function that $\Omega_1-\Omega_2$ produces is very steep on the domain $-1\leq x \leq 1$.  $\Omega_1-\Omega_2 = 3x^2-7.9x+1.75$.  Secant method fails with large slopes around the root because the method uses the slope between points on the function, which will not differ that much if the slope is already near vertical, so the method will hardly gain effeciency at each step.  Newton's method or Bisection method would better solve a function like this.

\section{Convergence, or the lack thereof}

\begin{enumerate}

\item $x_0=2$, first criteria

\begin{footnotesize}
\begin{spacing}{-.8}
Iterations: 30, calculated value: 403.4287934925369, actual value: 403.428793492735
\end{spacing}
\end{footnotesize}

\item $x_0=-2$, first criteria

\begin{footnotesize}
\begin{spacing}{-.8}
Iterations: 29, calculated value: 0.002478751478959605, actual value: 0.0024787521766663594
\end{spacing}
\end{footnotesize}

\item $x_0=-12$, first criteria

\begin{footnotesize}
\begin{spacing}{-.8}
Iterations: 100, calculated value: 0.05229714454636593, actual value: 2.319522830243574e-16
\end{spacing}
\end{footnotesize}

\item $x_0=2$, second criteria

\begin{footnotesize}
\begin{spacing}{-.8}
Iterations: 36, calculated value: 403.4287934927351, actual value: 403.428793492735
\end{spacing}
\end{footnotesize}
\vspace{.25in}
Note that the answer is accurate to all decimal places that were outputted.

\item $x_0=-2$, second criteria

\begin{footnotesize}
\begin{spacing}{-.8}
Iterations: 36, calculated value: 0.0024787521766719985, actual value: 0.0024787521766663594
\end{spacing}
\end{footnotesize}
\vspace{.25in}
This was accurate to 13 decimal places of accuracy.

\item $x_0=-12$, second criteria

\begin{footnotesize}
\begin{spacing}{-.8}
Iterations: 122, calculated value: 0.05109305371814203, actual value: 2.319522830243574e-16
\end{spacing}
\end{footnotesize}
\vspace{.25in}
This was accurate to 1 decimal place of accuracy.

\item $x_0=20$, second criteria

\begin{footnotesize}
\begin{spacing}{-.8}
Iterations: 188, calculated value: 1.1420073898156829e+26, actual value: 1.1420073898156806e+26
\end{spacing}
\end{footnotesize}

\item $x_0=-20$, second criteria

\begin{footnotesize}
\begin{spacing}{-.8}
Iterations: 188, calculated value: -1315533380.7076263, actual value: 8.756510762696549e-27
\end{spacing}
\end{footnotesize}
\vspace{.25in}
The summation was accurate for positive x values, but inaccurate for negative x values (and very innacurate for large negative x values).  This is because the series diverges due to the negative beign within the power, so the terms being added switch off being positive and negative, so the series gets further from the true answer (although for small x-values, it does not throw off the series as much, so we can still get results to a certain degree of accuracy.

\end{enumerate}

\section{Newton's method}

\begin{footnotesize}
\begin{spacing}{-.8}
final v: 0.016934795162778068, final w: 2.191866669273077, number of iterations: 5
\end{spacing}
\end{footnotesize}

\pagebreak

\section{Blackbody radiation}

\begin{footnotesize}
\begin{spacing}{-.8}
xold: 4.0, xnew: 3.145551586016073, y value: -8.919768895944435e-20\\
6.982509981073105e-19
xold: 3.145551586016073, xnew: 2.5801531082601983, y value: 7.938790872291784e-20\\
6.982509981073105e-19
xold: 2.5801531082601983, xnew: 2.8464024401202717, y value: -7.610525634677855e-21\\
6.982509981073105e-19
xold: 2.8464024401202717, xnew: 2.8231112466603054, y value: -5.134205172949907e-22\\
6.982509981073105e-19
xold: 2.8231112466603054, xnew: 2.8214263094363456, y value: 4.01354624142499e-24\\
6.982509981073105e-19
xold: 2.8214263094363456, xnew: 2.8214393788763745, y value: -2.075267602873835e-27\\
6.982509981073105e-19
xold: 2.8214393788763745, xnew: 2.8214393721221063, y value: -8.42350406614243e-33\\
6.982509981073105e-19
xold: 2.8214393721221063, xnew: 2.8214393721220787, y value: 3.9546967446679624e-35\\
Final y value: 3.9546967446679624e-35\\
Final x value: 2.8214393721220787\\
Number of iterations: 8\\
\end{spacing}
\end{footnotesize}

\end{document}